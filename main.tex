\documentclass{article}
\usepackage[utf8]{inputenc}

\title{Spadek swobodny}
\author{Adrian Śliwa}
\date{December 2022}
\usepackage[utf8]{inputenc}
\usepackage{polski}
\usepackage{graphicx}
\usepackage{amsmath}
\usepackage{tabularx}
\usepackage[polish]{babel}
\begin{document}

\maketitle

\tableofcontents

\newpage
\section{Wstęp teoretyczny}
„\textbf{Spadek swobodny} – ruch odbywający się wyłącznie pod wpływem ciężaru (siły grawitacji), bez
oporów ośrodka.
Przykłady:
\begin{itemize}
    \item  ruch planet wokół Słońca, ruch Księżyca wokół Ziemi,
    \item ruch statku kosmicznego z wyłączonym napędem,
    \item spadek masywnego ciała w pobliżu powierzchni Ziemi z niewielkiej wysokości (wówczas prędkość
spadku jest niewielka i siły oporu powietrza są zaniedbywalnie małe).
\end{itemize}
Przyjmuje się, że spadek rozpoczyna się od spoczynku, w odróżnieniu od ruchu w polu grawitacyjnym
z prędkością początkową zwanego rzutem.”
Wzór spadku swobodnego:
\begin{equation}
h(t) = x_0 - \frac{gt^2}{2}
\end{equation}

Zjawisko spadku swobodnego ilustruje rysunek \ref{fig:spadek_swobodny}.

\begin{figure}[hbt!]
    \centering
    \includegraphics[scale=0.5]{rys0010.jpg}
    \caption{Spadek swobodny}
    \label{fig:spadek_swobodny}
\end{figure}

\newpage

\section{Opis eksperymentów}
Jak widać na rysunku Rysunek \ref{fig:zjawisko} kulę zrzucono w dół.
\begin{figure}[hbt!]
    \centering
    \includegraphics{spadek2.png}
    \caption{zjawisko spadku swobodnego}
    \label{fig:zjawisko}
\end{figure}

\newpage
\section{Wyniki pomiarów}
Wyniki pomiarów można zobaczyć na tabeli \ref{Tab:pomiary}.


\begin{table}[hbt!]
\caption{wyniki pomiarów}
\label{Tab:pomiary}
\begin{center}

\begin{tabularx}{0.8\textwidth} { 
  | >{\centering\arraybackslash}X 
  | >{\centering\arraybackslash}X 
  | >{\centering\arraybackslash}X | }
\hline
Lp. & T[s] & S[m] \\
\hline
1 & 0 & 0 \\
2 & 0 & 0 \\
3 & 0 & 0 \\
4 & 0 & 0 \\
5 & 0 & 0 \\
6 & 0 & 0 \\
7 & 0 & 0 \\
8 & 0 & 0 \\
9 & 0 & 0 \\
10 & 0 & 0 \\
11 & 0 & 0 \\
12 & 0 & 0 \\
13 & 0 & 0 \\
14 & 0 & 0 \\
15 & 0 & 0 \\

\hline
\end{tabularx}
\end{center}
\end{table}


\section{Wnioski}
Wyniki pomiarów zgadzają się ze wzorami, zatem teoria jest prawdziwa.

\end{document}
