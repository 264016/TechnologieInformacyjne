\documentclass{article}
\usepackage[utf8]{inputenc}

\title{Spadek swobodny}
\author{Adrian Śliwa}
\date{December 2022}
\usepackage[utf8]{inputenc}
\usepackage{polski}
\usepackage{graphicx}
\usepackage{amsmath}
\usepackage{pgfplotstable,filecontents}
\usepackage{csvsimple}
\usepackage{tabularx}
\usepackage{pgfplots}
\usepackage[polish]{babel}

\begin{document}


\maketitle

\tableofcontents

\newpage
\section{Wstęp teoretyczny}
„\textbf{Spadek swobodny} – ruch odbywający się wyłącznie pod wpływem ciężaru (siły grawitacji), bez
oporów ośrodka.
Przykłady:
\begin{itemize}
    \item  ruch planet wokół Słońca, ruch Księżyca wokół Ziemi,
    \item ruch statku kosmicznego z wyłączonym napędem,
    \item spadek masywnego ciała w pobliżu powierzchni Ziemi z niewielkiej wysokości (wówczas prędkość
spadku jest niewielka i siły oporu powietrza są zaniedbywalnie małe).
\end{itemize}
Przyjmuje się, że spadek rozpoczyna się od spoczynku, w odróżnieniu od ruchu w polu grawitacyjnym
z prędkością początkową zwanego rzutem.”
Wzór spadku swobodnego:
\begin{equation}
h(t) = x_0 - \frac{gt^2}{2}
\end{equation}

Zjawisko spadku swobodnego ilustruje rysunek \ref{fig:spadek_swobodny}.

\begin{figure}[hbt!]
    \centering
    \includegraphics[scale=0.5]{rys0010.jpg}
    \caption{Spadek swobodny}
    \label{fig:spadek_swobodny}
\end{figure}

\newpage

\section{Opis eksperymentów}
Jak widać na rysunku Rysunek \ref{fig:zjawisko} kulę zrzucono w dół.
\begin{figure}[hbt!]
    \centering
    \includegraphics{spadek2.png}
    \caption{zjawisko spadku swobodnego}
    \label{fig:zjawisko}
\end{figure}

\newpage
\section{Wyniki pomiarów}
Wyniki pomiarów można zobaczyć na tabeli \ref{Tab:pomiary}.

%\pgfplotstabletypeset[
 %    columns={k1,k4,k2,k3},
 %   ]{data4.csv}

\begin{table}[hbt!]
\caption{wyniki pomiarów}
\label{Tab:pomiary}
\begin{center}

\begin{tabularx}{0.8\textwidth} { 
  | >{\centering\arraybackslash}X 
  | >{\centering\arraybackslash}X 
  | >{\centering\arraybackslash}X | }
\hline
Lp. & T[s] & S[m] \\
\hline
1 & 0 & 10 \\
2 & 1 & 10 \\
3 & 2 & 01 \\
4 & 3 & 01 \\
5 & 4 & 01 \\
6 & 5 & 04 \\
7 & 6 & 04 \\
8 & 7 & 50 \\
9 & 8 & 50 \\
10 & 9 & 60 \\
11 & 10 & 60 \\
12 & 20 & 07 \\
13 & 30 & 80 \\
14 & 40 & 90 \\
15 & 50 & 10 \\



\hline
\end{tabularx}
\end{center}
\end{table}
\begin{filecontents*}{data3.csv}
\end{filecontents*}



\begin{center}
    
\begin{tikzpicture}[yscale=1.65, xscale=1.65]
\begin{axis}[
    title={Spadek swobodny},
    xlabel={t[s]},
    ylabel={s[m]},
    legend pos=north west,
    ymajorgrids=true,
    grid style=dashed,
        enlargelimits=false,
]
\addplot [
    domain=-0.5:10.5, 
    samples=100, 
    color=red,
]
{9.80665 * x^2 /2};
\addlegendentry{\(9.80665 * x^2 /2\)}
\addplot+[
    only marks,
    scatter,
    mark=square,
    mark size=1.0pt]
table[meta=k4]
{data4.csv};


\end{axis}
\end{tikzpicture}


\end{center}



\section{Wnioski}
Wyniki pomiarów zgadzają się ze wzorami, zatem teoria jest prawdziwa.

\end{document}
